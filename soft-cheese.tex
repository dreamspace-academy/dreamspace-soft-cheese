\documentclass[12pt]{article}

\usepackage{sbc-template} 
\usepackage{graphicx,url}
\usepackage{url}
\usepackage[english]{babel} 
\usepackage[utf8]{inputenc} 
\usepackage[T1]{fontenc}
\usepackage[normalem]{ulem}
\usepackage[hidelinks]{hyperref}

\usepackage[square,authoryear]{natbib}
\usepackage{amssymb} 
\usepackage{mathalfa} 
\usepackage{algorithm} 
\usepackage{algpseudocode} 
\usepackage[table]{xcolor}
\usepackage{array}
\usepackage{titlesec}
\usepackage{mdframed}
\usepackage{listings}

\usepackage{amsmath} 
\usepackage{booktabs}

\urlstyle{same}

\newcolumntype{L}[1]{>{\raggedright\let\newline\\\arraybackslash\hspace{0pt}}m{#1}}
\newcolumntype{C}[1]{>{\centering\let\newline\\\arraybackslash\hspace{0pt}}m{#1}}
\newcolumntype{R}[1]{>{\raggedleft\let\newline\\\arraybackslash\hspace{0pt}}m{#1}}

\newcommand\Tstrut{\rule{0pt}{2.6ex}} 
\newcommand\Bstrut{\rule[-0.9ex]{0pt}{0pt}} 
\newcommand{\scell}[2][c]{\begin{tabular}[#1]{@{}c@{}}#2\end{tabular}}

\usepackage[nolist,nohyperlinks]{acronym}

\title{Soft Cheesemaking in Batticaloa at DreamSpace Bio Lab}

\author{Cristian Silva\inst{1} Ahac Maden\inst{1} Pramodya Saumyamali \inst{2}} 


\address{Digital Naturalism Conference - 2022
	\email{cristiansilvaofficial@gmail.com}
}



\begin{document} 
	
	\maketitle
	
	\begin{abstract}
	     Cheese is a fermented dairy product and the practice of making cheese goes back 8000 to 10000 years since the initial domestication of milk-producing animals took place. Cheese is an excellent source of protein, vitamins and calcium. Cheese products have a higher amount of calcium than milk. Apart from having higher calcium content than milk, cheese can be a better source of calcium for many people because is low in lactose compared to  milk. In this project, we explored the process of making cheese with low tech, and low-budget settings. Which could be beneficial for underserved local communities in Sri Lanka.
	\end{abstract}
	    
		
	
	
	\section{Introduction}
	\label{sec:introduction}
	Cheese is of various types and the most common cheese types are soft cheese and spread cheese. Soft cheese is any young or fresh cheese with a high moisture content that makes it soft enough to spread and they are unripened cheese made by coagulating casein (milk proteins) with acid. Cheese spread is prepared using one or more cheeses or processed cheese and sometimes additional ingredients such as vegetables, fruits, meats and various spices and seasonings. These cheese types are made from the milk of cows, buffaloes, goats and sheep and formed by coagulation of casein protein in them. Cheese is considered a source of saturated fatty acids which is good for health and cheese is valued for its portability, long shelf life, and high content of fat, protein, calcium and phosphorus.

	\section{Objective}
	\label{sec:objective}
	
Establish know-how on making soft cheese with easily available ingredients in Batticaloa, Sri Lanka.



	
	
	




	
	
	\section{Materials}
	\label{sec:materials}
	
	Cow milk, Buffalo milk, Lime, Chewable tablets of papain enzyme, Fannel clothes, Stove, Strainer, Metal cooking pot

	
	\section{Metodology}
	\label{sec:metodology}
	
	Pour the milk to a metal pot and heat the milk,  bringing the milk to an almost boiling stage. Cut limes into two and get the lime juice out of the limes. Once the milk is about to boil, pour the lime juice and stir the milk while it is coagulating. Once the milk is coagulated leave the mixture for 10 minutes to settle down. Filter out the cheese whey by using a fannel cloth. Once the whey is removed from the cheese, wrap it in a flannel cloth and store it in the refrigerator and consume the cheese within 3 to 4 days.

 
\subsection{Preparation of  Paneer with Cow and Buffalo milk and Lime}

\subsubsection{Batch 01 - Cow milk paneer
}
 
 While making the first batch, the above-mentioned protocol was followed and 6 limes were used to get the lime juice. Two litres of cow milk were used. 

    \subsubsection{ Buffalo milk paneer}


While making the second batch, the above-mentioned protocol was followed and 20 limes were used to get the lime juice. 3.5  liters of cow milk was used. 

\subsubsection{Buffalo + Cow milk paneer }
While making the third batch, the above-mentioned protocol was followed and 20 limes were used to get the lime juice and three tamarinds were added. 1.9 litres of cow milk and 1.6 L of buffalo milk were used for this batch

\subsection{Preparation of  Paneer with Cow milk,  Lime and chewable papain tablets}

\subsubsection{ Cow milk + lime + papain chewable tablets}

While making the third batch, the above-mentioned protocol was followed with mild alterations. 06 limes were used to get the lime juice and 12 tablets of chewable papain enzyme were drained and the powder of the tablets was added along with lime to 2 L of cow milk. 

        \section{Observations}

        Totally 4 batches were made with cow milk or buffalo milk.
        
        Batch 1: Milk was coagulated slowly at the beginning. However, after 5 minutes of stirring milk started coagulating faster than before.


        Batch 2: Final texture was not like Mozzarella cheese and a sudden, proper coagulation was not observed.

        Batch 3: Coagulation was formed effectively and efficiently. The texture was observed pulp like and clumsy compared to other batches.

        Batch 4: The texture was observed more different than other batches and it was more stingy. The time taken for coagulation was comparatively higher and the taste was sour compared to other batches.


        \begin{figure}[ht]
\centering
\includegraphics[width=0.7\linewidth]{figures/cheese.JPG}
\caption{Soft cheese made with lime juice and milk}
\label{fig:view}
\end{figure}


        \section{Discussion}

        Cheese provides a valuable alternative to fermented milk and yoghurts as a food vehicle for probiotic delivery, due to certain potential advantages. It creates a buffer against the highly acidic environment in the gastrointestinal tract and thus creates a more favourable environment for probiotic survival throughout the gastric transit, due to higher pH. Moreover, the dense matrix and relatively high-fat content of cheese may offer additional protection to probiotic bacteria in the stomach.

The colloid of milk contains two major proteins, casein and whey protein. The important source of protein to make cheese in the milk is the casein component, which the form of casein micelle can be coagulated by adjusting the pH of the milk. But, this coagulation process is mainly generated by the surface interaction between micelles which is caused by the surface charge of micelles near to isoelectric charge. So, the formation of cheese products by pH control is considered to have no good texture. Based on the enzymatic reaction of casein protein molecules, the coagulation process of milk to cheese can be preceded by a coagulation agent of a specific enzyme. The performance of the papain enzyme can be used to coagulate casein micelle in the appropriate condition. The milk clotting time is inversely proportional to the amount of coagulant incorporated. The lowest amount of coagulant used gave a cheese with a flavour and colour better than cheeses obtained with a significant amount of coagulant. However, they are better in terms of their texture and firmness. 

When the pH is in the range of 6.0–6.7  the coagulation of milk is faster, and the firmness of cheese is higher. The cooking temperature in the cheese vat influences moisture removal from the curd during cheese making. Differences in moisture content could have a significant impact on soft cheese functionality. Thus, differences in cooking temperature may affect the chemical composition of cheese which may cause differences in proteolysis and functional properties of soft cheese.

In the first batch, the milk was coagulated but at the initial stage, it was difficult because we didn’t know when to add lime to the milk and the team figured out its best to add lime juice before the milk reaches the boiling stage while heating. In the preparation of the second batch, the final texture was not like the Mozzarella which the team expected to have. However, it rapidly coagulated when lime juice was added.  The third batch was formed well and the coagulation was faster and more efficient. The texture was pulp-like and coagulated faster than the other three batches. In the fourth batch, the texture was different and felt and looked  much more like sticking milk powder compared to other batches. Further, the taste had a sourer taste than other batches. The reason for the different texture and taste is probably the other ingredients present in chewable papain tablets.


In the next iterations of this project, every ingredient needs to be measured precisely and the temperature and pH of the milk should be measured at each step of the process. Measuring the milk density could be useful to gain more insights.  Once the cheese is made, the weight of each cheese batch should  be recorded. The weight of the milk used to make cheese needs to be measured before making cheese.  In order to figure out how to effectively coagulate most milk casein into cheese, recording the weight and density of cheese and milk will be helpful. Further, experiments are needed to explore the possibility of using milk whey to make other food products.        	
    \end{document}

